% !TeX encoding = UTF-8
% !TeX spellcheck = uk_UA

\documentclass[12pt]{article}

\usepackage[utf8]{inputenc}
\usepackage[T2A]{fontenc}
\usepackage[ukrainian]{babel}
\usepackage[margin=2.5cm]{geometry}
\usepackage{enumitem}
\usepackage{amsfonts,amssymb,amsmath,amsthm}
\usepackage[all]{xy}

\usepackage[unicode]{hyperref}
\hypersetup{
    colorlinks=true,
    citecolor=black,
    filecolor=black,
    linkcolor=black,
    urlcolor=black
}

\usepackage{enumitem}
\usepackage{multicol}
\usepackage[usenames,dvipsnames,svgnames,table]{xcolor}
\newcommand\bZ{\mathbb{Z}}
\newcommand\bR{\mathbb{R}}
\newcommand\bN{\mathbb{N}}

% %\usepackage{makeidx}
% \usepackage{imakeidx}
% \makeindex[program=texindy,options=-M mystyle.xdy]
% %\makeindex

\newtheorem{definition}[subsection]{Означення}
\newtheorem{zad-def}[subsection]{\color{RoyalPurple}Задача-означення}
\newtheorem{zadacha}[subsection]{\color{RoyalPurple}Задача}
\newcommand\zad[1]{\begin{zadacha}\rm#1\end{zadacha}}
\newcommand\zadl[2]{\begin{zadacha}\label{#1}\rm#2\end{zadacha}}
\newcommand\zaddef[1]{\begin{zad-def}\rm#1\end{zad-def}}
\newcommand\zaddefl[2]{\begin{zad-def}\label{#1}\rm#2\end{zad-def}}

\newcommand\dyv[1]{Див. також задачі #1.}

% -------- begin myemph --------------------
\makeatletter
\newcommand\testshape{family=\f@family; series=\f@series; shape=\f@shape.}
\def\myemphInternal#1{\if n\f@shape%
\begingroup\itshape #1\endgroup\/%
\else\begingroup\bfseries #1\endgroup%
\fi}
\def\myemph{\futurelet\testchar\MaybeOptArgmyemph}
\def\MaybeOptArgmyemph{\ifx[\testchar \let\next\OptArgmyemph
                 \else \let\next\NoOptArgmyemph \fi \next}
\def\OptArgmyemph[#1]#2{\index{#1}\myemphInternal{#2}}
\def\NoOptArgmyemph#1{\myemphInternal{#1}}
\makeatother
% -------- end myemph --------------------
\newcommand\toindex[1]{\myemph[#1]{~\!\!}}


\newcommand\bemph[1]{{\it\bfseries#1}}

\newcommand\id{\mathrm{id}}          % identity map    
\newcommand\cl[1]{\overline{#1}}     % closure
\newcommand\CL{\mathrm{cl}}          % Kuratowski operator
\newcommand\Int[1]{\mathrm{Int}\left(#1\right)}     % closure
\newcommand\Ext[1]{\mathrm{Ext}\left(#1\right)}     % closure
\newcommand\Fr[1]{\mathrm{Fr}\left(#1\right)}     % closure


\title{Вступ в теорію груп}
\author{Сергій Максименко}
\begin{document}
\maketitle
% \tableofcontents

\section{Поняття групи}

\zad{
Описати множини всіх симетрій 
\begin{multicols}{2}
\begin{enumerate}[label=\arabic*), itemsep=1ex]
\item правильного трикутника;
\item правильного квадрата;
\item правильного п'ятикутника;
\item правильного шестикутника (на лекції я сказав ось неправильне про шестикутник. Знайдіть помилку);
\item правильного n-кутника;
\item прямокутника;
\item ромба;
\item паралелограма;
\item кола;
\item рівнобедреного трикутника
\end{enumerate}
\end{multicols}
}

\zad{
Описати симетрії букв українського та англійського алфавіту.
Чи є букви у яких множини симетрій <<схожі>>, якщо так, то чим? 
}

\zad{
Чим <<схожі>> чи <<однакові>> множини симетрій правильного трикутника, букви Y, символу <<Мерседес>>?
}


\zad{
Які симетрії є у числової прямої?
}

\zad{
Які з наступних функцій (можливо їх графіки) є <<симетричними>> і в якому сенсі?
Якщо так, то опишіть множини симетрій цих функцій.
\begin{multicols}{2}
\begin{enumerate}[label=\arabic*)]
\item $f(x)=\sin(x)$
\item $f(x)=\cos(x)$
\item $f(x)=\tg(x)$
\item $f(x)=\ctg(x)$
\item $f(x)=x^2$
\item $f(x)=x^3$
\item $f(x)=|x|$
\item $f(x)=2 x + 4$
\item $f(x) = x^2 + 5x + 6$
\item $f(x) = 2^x$
\item періодична функція з періодом $5$
\end{enumerate}
\end{multicols}
}


\zad{
Перевірити, що множина цілих чисел 
\[ 
    \bZ = \{ \ldots, -3, -2, -1, 0, 1, 2, \ldots \} 
\]
утворює групу відносно одерації додавання.
Тобто відображення 
\[
\mu: \bZ\times\bZ \to \bZ, \qquad \mu(x,y) =x+y,
\]
задовольняє аксіоми групи.
Що буде нейтральним елементом $\bZ$?
Що є оберненим елементом для $x\in\bZ$?
}


\zad{
Чи утворює групу множина натуральних чисел $\bN = \{ 1,2,3,\ldots\}$ відносно такої ж операції додавання:
\[
\mu: \bN\times\bN \to \bN, \qquad \mu(x,y) =x+y ?
\]
}

\zad{
Встановити, які з операцій на множинах задовольняють аксіоми груп:
\begin{enumerate}
\item
$\bR$ -- множина дійсних чисел з операцією додавання $\mu: \bR\times\bR \to \bR$, $\mu(x,y) =x+y$;

\item
$\bR$ -- множина дійсних чисел з операцією множення $\mu: \bR\times\bR \to \bR$, $\mu(x,y) =xy$;

\item
$\bR$ -- множина дійсних чисел з операцією віднімання $\mu: \bR\times\bR \to \bR$, $\mu(x,y) =x-y$;

\end{enumerate}
}




\end{document}
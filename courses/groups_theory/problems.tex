\section{Поняття групи}

\zad{
Описати множини всіх симетрій 
\begin{multicols}{2}
\begin{enumerate}[label=\arabic*), itemsep=1ex]
\item правильного трикутника;
\item правильного квадрата;
\item правильного п'ятикутника;
\item правильного шестикутника (на лекції я сказав ось неправильне про шестикутник. Знайдіть помилку);
\item правильного n-кутника;
\item прямокутника;
\item ромба;
\item паралелограма;
\item кола;
\item рівнобедреного трикутника
\end{enumerate}
\end{multicols}
}

\zad{
Описати симетрії букв українського та англійського алфавіту.
Чи є букви у яких множини симетрій <<схожі>>, якщо так, то чим? 
}

\zad{
Чим <<схожі>> чи <<однакові>> множини симетрій правильного трикутника, букви Y, символу <<Мерседес>>?
}


\zad{
Які симетрії є у числової прямої?
}

\zad{
Які з наступних функцій (можливо їх графіки) є <<симетричними>> і в якому сенсі?
Якщо так, то опишіть множини симетрій цих функцій.
\begin{multicols}{2}
\begin{enumerate}[label=\arabic*)]
\item $f(x)=\sin(x)$
\item $f(x)=\cos(x)$
\item $f(x)=\tg(x)$
\item $f(x)=\ctg(x)$
\item $f(x)=x^2$
\item $f(x)=x^3$
\item $f(x)=|x|$
\item $f(x)=2 x + 4$
\item $f(x) = x^2 + 5x + 6$
\item $f(x) = 2^x$
\item періодична функція з періодом $5$
\end{enumerate}
\end{multicols}
}


\zad{
Перевірити, що множина цілих чисел 
\[ 
    \bZ = \{ \ldots, -3, -2, -1, 0, 1, 2, \ldots \} 
\]
утворює групу відносно одерації додавання.
Тобто відображення 
\[
\mu: \bZ\times\bZ \to \bZ, \qquad \mu(x,y) =x+y,
\]
задовольняє аксіоми групи.
Що буде нейтральним елементом $\bZ$?
Що є оберненим елементом для $x\in\bZ$?
}


\zad{
Чи утворює групу множина натуральних чисел $\bN = \{ 1,2,3,\ldots\}$ відносно такої ж операції додавання:
\[
\mu: \bN\times\bN \to \bN, \qquad \mu(x,y) =x+y ?
\]
}

\zad{
Встановити, які з операцій на множинах задовольняють аксіоми груп:
\begin{enumerate}
\item
$\bR$ -- множина дійсних чисел з операцією додавання $\mu: \bR\times\bR \to \bR$, $\mu(x,y) =x+y$;

\item
$\bR$ -- множина дійсних чисел з операцією множення $\mu: \bR\times\bR \to \bR$, $\mu(x,y) =xy$;

\item
$\bR$ -- множина дійсних чисел з операцією віднімання $\mu: \bR\times\bR \to \bR$, $\mu(x,y) =x-y$;

\end{enumerate}
}


\section{Ізоморфізми груп}

Нехай $f:A \to B$ -- відображення між групами $(A, *)$ та $(B,\star)$ з операціями $*$ та $\star$ відповідно.
Воно називається \myemph{ізоморфізмом груп}, якщо
\begin{enumerate}
\item $f$ -- бієкція
\item для довільних $x,y\in A$ виконується співвідношеня: $f(x*y) = f(x) \star f(y)$.
\end{enumerate}

Ізоморфізм $f:A \to A$ групи $A$ на себе називається \myemph{автоморфізмом групи}.
Множина всіх автоморфізмів групи на себе позначається через $\Aut(A)$.


% Нехай $\bR_{+} = (0,\infty)$ -- множина додатних чисел.


\zad{
Довести, що відображення $f:\bR \to \bR_{+}$, $f(x) = 3^x$, є ізоморфізмом групи $\bR$ з операцією додавання чисел на групу $\bR_{+}$ з операцією множення.
}


\zad{
Довести, що відображення $f:\bR_{+} \to \bR$, $f(x) = \log_{7} x$, є ізоморфізмом групи $(\bR_{+}, \cdot)$ на групу $(\bR, +)$.
}


\zad{
Нехай $f: A \to B$ -- ізоморфізм груп $A$ і $B$.
Довести, що обернене відображення $f^{-1}: B \to A$ також ізоморфізм.
}


\zad{
Нехай $f: A \to B$, $g:B \to C$ -- ізоморфізми груп.
Довести, що композиція $g \circ f: A \to C$ також ізоморфізм груп.
}


\zad{
Позначимо через $\Aut(A)$ -- множина всіх ізоморфізмів групи $A$ на себе.
Довести, що $\Aut(A)$ є групою відносно композиції ізоморфізмів як відображень.
}


\zad{
Описати всі автоморфізми групи цілих чисел $(\bZ,+)$.
}



\zad{
Описати всі підмножини в групі цілих чисел $\bZ$ відносно додавання, які також є утворюють групи відносно додавання.
}


\zad{
Нехай $4\bZ$ і $7\bZ$ множини цілих чисел кратних 4 та 7 вдповідно.
Перевірити, що вони уворюють групи відносно операції додавання.
Чи ізоморфні ці групи?
Якщо так, побудувати всі можливі ізоморфізми між $(4\bZ,+)$ і $(7\bZ,+)$.
}

